\documentclass[11pt]{article}
	
%%%%%%%%%%%%%%%%%%%%%%%%%%%%%%%%%%%%%%%%%%%%%%%%%%%%%%%%%%%%%%%%%%%%%%
%\pdfminorversion=4
% NOTE: To produce blinded version, replace "0" with "1" below.
\newcommand{\blind}{0}

%%%%%%% IISE Transactions margin specifications %%%%%%%%%%%%%%%%%%%
% DON'T change margins - should be 1 inch all around.
\addtolength{\oddsidemargin}{-.5in}%
\addtolength{\evensidemargin}{-.5in}%
\addtolength{\textwidth}{1in}%
\addtolength{\textheight}{1.3in}%
\addtolength{\topmargin}{-.8in}%
\makeatletter
\renewcommand\section{\@startsection {section}{1}{\z@}%
                                   {-3.5ex \@plus -1ex \@minus -.2ex}%
                                   {2.3ex \@plus.2ex}%
                                   {\normalfont\fontfamily{phv}\fontsize{16}{19}\bfseries}}
\renewcommand\subsection{\@startsection{subsection}{2}{\z@}%
                                     {-3.25ex\@plus -1ex \@minus -.2ex}%
                                     {1.5ex \@plus .2ex}%
                                     {\normalfont\fontfamily{phv}\fontsize{14}{17}\bfseries}}
\renewcommand\subsubsection{\@startsection{subsubsection}{3}{\z@}%
                                    {-3.25ex\@plus -1ex \@minus -.2ex}%
                                     {1.5ex \@plus .2ex}%
                                     {\normalfont\normalsize\fontfamily{phv}\fontsize{14}{17}\selectfont}}
\makeatother
%%%%%%%%%%%%%%%%%%%%%%%%%%%%%%%%%%%%%%%%%%%%%%%%%%%%%%%%%%%%%%%%%%%%%%%%%

%%%%% IISE Transactions package list %%%%%%%%%%%%%%%%%%%%%%%%%%%%%%%%%%%%%%
\usepackage{amsmath}
\usepackage{graphicx}
\usepackage{enumerate}
\usepackage{natbib} %comment out if you do not have the package
\usepackage{url} % not crucial - just used below for the URL
%%%%%%%%%%%%%%%%%%%%%%%%%%%%%%%%%%%%%%%%%%%%%%%%%%%%%%%%%%%%%%%%%%%%%%%

%%%%% Author package list and commands %%%%%%%%%%%%%%%%%%%%%%%%%%%%%%%%%%%%%%%%%%%%%
%%%%% Here are some examples %%%%%%%%%%%%%%
%	\usepackage{amsfonts, amsthm, latexsym, amssymb}
%	\usepackage{lineno}
%	\newcommand{\mb}{\mathbf}
%%%%%%%%%%%%%%%%%%%%%%%%%%%%%%%%%%%%%%%%%%%%%%%%%%%%%%%%%%%%%%%%%%%%%%%%%%%%%%

\begin{document}
	
		%%%%%%%%%%%%%%%%%%%%%%%%%%%%%%%%%%%%%%%%%%%%%%%%%%%%%%%%%%%%%%%%%%%%%%%%%%%%%%
	\def\spacingset#1{\renewcommand{\baselinestretch}%
		{#1}\small\normalsize} \spacingset{1}
	%%%%%%%%%%%%%%%%%%%%%%%%%%%%%%%%%%%%%%%%%%%%%%%%%%%%%%%%%%%%%%%%%%%%%%%%%%%%%%
	
	\if0\blind
	{
		\title{\bf Simulation-guided Data Assimilation for Kinetics Modeling of Potassium Channel Isoforms in Mouse Cardiomyocytes}
		\author{Haedong Kim $^a$, Hui Yang $^a$, Andrew R. Ednie $^b$, and Eric S. Bennett $^b$ \\
		$^a$ Department of Industrial and Manufacturing Engineering, \\
		The Pennsylvania State University, State College, PA USA \\
        $^b$ Department of Neuroscience, Cell Biology, and Physiology, \\
        Wright State University, City, OH USA}
		\date{}
		\maketitle
	} \fi
	
	\if1\blind
	{
        \title{\bf \emph{IISE Transactions} \LaTeX \ Template}
		\author{Author information is purposely removed for double-blind review}
		
\bigskip
		\bigskip
		\bigskip
		\begin{center}
			{\LARGE\bf \emph{IISE Transactions} \LaTeX \ Template}
		\end{center}
		\medskip
	} \fi
	\bigskip
		
\begin{abstract}
Abstract goes here.
\end{abstract}
		
\noindent%
{\it Keywords:} \emph{IISE Transactions}; \LaTeX; Manuscript format; Taylor \& Francis.

%\newpage
\spacingset{1.5} % DON'T change the spacing!


\section{Introduction} \label{s:intro}
\begin{itemize}
    \item \textbf{Motivation} Voltage-gated ion channels (VGICs) are the functional unit of the electrical conduction system of the heart that controls the synchronous contraction of numerous cardiomyocytes. Among major cardiac VGICs, K\textsuperscript{+} channels (K\textsubscript{v}) have a distinctive feature: they have various isoforms that contribute to K\textsuperscript{+} current, while the dominant isoform is responsible for Na\textsuperscript{+} and Ca\textsuperscript{2+} currents. Although each K\textsubscript{+} generated through K\textsubscript{v} isoform has a different role in the repolarization phase, only the sum of these individual K\textsuperscript{+} currents (I\textsubscript{Ksum}) can be recorded through in-vitro experiments. Data analysis tools are used to estimate information about K\textsubscript{v} isoforms from experimental I\textsubscript{Ksum} recordings.
    \item \textbf{Gaps \& Needs} A traditional approach is based on the data-driven curve fitting method that assumes the functional forms for the shape of individual K\textsuperscript{+} currents. This method directly fits individual K\textsuperscript{+} currents to experimental I\textsubscript{Ksum} recordings by searching the shape parameters of the current functions that have the best fit. It has been proven that the curve fitting approach describes I\textsubscript{Ksum} well with estimated K\textsuperscript{+} currents. However, the mathematical form of this classical method only characterizes the shape of the currents, not the kinetics of corresponding K\textsubscript{v} isoforms. In addition, the curve fitting method handles experimental I\textsubscript{Ksum} recordings from the same cell independently, so it lacks the cellular-level analysis.
    \item \textbf{Objective} As opposed to the traditional data-driven curve fitting method, we propose a novel framework of biophysics-driven data assimilation method for kinetics modeling of K\textsubscript{v} isoforms based on computer models of K\textsuperscript{+} currents that can simulate underlying gating kinetics of the currents.
    \item \textbf{Approach} 
    \item \textbf{Experiment \& Performance} The proposed framework outperforms the curve fitting method for minimizing discrepancies between experimental and estimated total K\textsuperscript{+} currents. We also provide the kinetics modeling results that are not possible in the traditional method.
    \item \textbf{Significance} Experimental results show that the proposed method has strong potential to be a novel analysis tool to decompose I\textsubscript{Ksum} recordings to estimate not only the current characteristics but also kinetics of K\textsubscript{v} isoforms.
\end{itemize}

The review of literature can be a separate section but it can be merged with the introduction section. One can find a number of reference examples at the end of this template, like,
\begin{itemize}
\item A book: \cite{Hastie2009}.
\item A journal article: \cite{Byon2013}.
\item A conference paper: \cite{Breunig2000}.
\item A book chapter: \cite{Yang2010}.
\item A technical report: \cite{Kelley2007}.
\end{itemize}

action potential and various ioninc currents
\begin{figure}
    \centering
    \includegraphics{figs/ap_underlying_currents.pdf}
    \caption{The action potentials and underlying ionic currents in \textbf{(A)} human ventricular cardiomyocytes and \textbf{(B)} mouse ventricular cardiomyocytes \cite{nerbonne2003molecular}.}
    \label{fig:ap_underlying_currents}
\end{figure}

\section{Methods} \label{s:methods}

\subsection{\emph{Computer Models of Potassium Channel Isoforms}} \label{s:methods.1}
Various K\textsuperscript{+} currents play different roles in cardiac repolarization. 
\begin{figure}
    \centering
    \includegraphics{figs/biomarkers.pdf}
    \caption{Six biomarkers of K\textsuperscript{+} currents that are subjects of sensitivity analysis. (A) I\textsubscript{Kto}, (B) I\textsubscript{Kslow1}, (C) I\textsubscript{Kslow2}, (D) I\textsubscript{Kss}. All currents are simulated for the purpose of illustration and biomarker labels refer to a: at 10 ms, b: 25\%, c: 50\%, d: 75\%, e: peak, and f: time constant}
    \label{fig:biomarkers}
\end{figure}

\subsection{\emph{Sensitivity Analysis of Calibration Parameters}} \label{s:methods.2}
this section shows the results of the variable screening.
\begin{figure}
    \centering
    \includegraphics{figs/doe_kto.pdf}
    \caption{Half-normal plots I\textsubscript{Kto}}
    \label{fig:doe_kto}
\end{figure}

\begin{figure}
    \centering
    \includegraphics{figs/doe_kslow1.pdf}
    \caption{Half-normal plots I\textsubscript{Kslow1}}
    \label{fig:doe_kslow1}
\end{figure}

\begin{figure}
    \centering
    \includegraphics{figs/doe_kslow2.pdf}
    \caption{Half-normal plots I\textsubscript{Kslow2}}
    \label{fig:doe_kslow2}
\end{figure}

\begin{figure}
    \centering
    \includegraphics{figs/doe_kss.pdf}
    \caption{Half-normal plots I\textsubscript{Kss}}
    \label{fig:doe_kss}
\end{figure}

\subsection{\emph{Model Calibration}} \label{s:methods.3}
calibration procedure; non-linear optimization methods
\begin{figure}
    \centering
    \includegraphics{figs/pkto.pdf}
    \caption{parameter distribution}
    \label{fig:pkto}
\end{figure}

\begin{figure}
    \centering
    \includegraphics{figs/pkslow1.pdf}
    \caption{parameter distribution}
    \label{fig:pkslow1}
\end{figure}

\begin{figure}
    \centering
    \includegraphics{figs/pkslow2.pdf}
    \caption{parameter distribution}
    \label{fig:pkslow2}
\end{figure}

\begin{figure}
    \centering
    \includegraphics{figs/pkss.pdf}
    \caption{parameter distribution}
    \label{fig:pkss}
\end{figure}

\subsection{\emph{Graphical User Interface}} \label{s:methods.4}
show how to use it

\subsection{\emph{Case Study Glycosylations}} \label{s:methods.5}
case study

\section{Results} \label{s:results}
Perhaps numerical analysis of performance, comparison, and demonstration of applicability and impact using real data.

\section{Conclusion}\label{s:conclusion}
A paper ends with a conclusion or summary section.

\if0\blind{
\section*{Acknowledgements}
The authors acknowledge the generous support from the funding agency of XYZ.	} \fi

\bibliographystyle{chicago}
\spacingset{1}
\bibliography{IISE-Trans}
	
\end{document}
