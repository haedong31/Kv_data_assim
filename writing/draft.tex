\documentclass[11pt]{article}
	
%%%%%%%%%%%%%%%%%%%%%%%%%%%%%%%%%%%%%%%%%%%%%%%%%%%%%%%%%%%%%%%%%%%%%%
%\pdfminorversion=4
% NOTE: To produce blinded version, replace "0" with "1" below.
\newcommand{\blind}{0}

%%%%%%% IISE Transactions margin specifications %%%%%%%%%%%%%%%%%%%
% DON'T change margins - should be 1 inch all around.
\addtolength{\oddsidemargin}{-.5in}%
\addtolength{\evensidemargin}{-.5in}%
\addtolength{\textwidth}{1in}%
\addtolength{\textheight}{1.3in}%
\addtolength{\topmargin}{-.8in}%
\makeatletter
\renewcommand\section{\@startsection {section}{1}{\z@}%
                                   {-3.5ex \@plus -1ex \@minus -.2ex}%
                                   {2.3ex \@plus.2ex}%
                                   {\normalfont\fontfamily{phv}\fontsize{16}{19}\bfseries}}
\renewcommand\subsection{\@startsection{subsection}{2}{\z@}%
                                     {-3.25ex\@plus -1ex \@minus -.2ex}%
                                     {1.5ex \@plus .2ex}%
                                     {\normalfont\fontfamily{phv}\fontsize{14}{17}\bfseries}}
\renewcommand\subsubsection{\@startsection{subsubsection}{3}{\z@}%
                                    {-3.25ex\@plus -1ex \@minus -.2ex}%
                                     {1.5ex \@plus .2ex}%
                                     {\normalfont\normalsize\fontfamily{phv}\fontsize{14}{17}\selectfont}}
\makeatother
%%%%%%%%%%%%%%%%%%%%%%%%%%%%%%%%%%%%%%%%%%%%%%%%%%%%%%%%%%%%%%%%%%%%%%%%%

%%%%% IISE Transactions package list %%%%%%%%%%%%%%%%%%%%%%%%%%%%%%%%%%%%%%
\usepackage{amsmath}
\usepackage{amsfonts}
\usepackage{mathtools}
\usepackage{graphicx}
\usepackage{enumerate}
\usepackage{natbib} %comment out if you do not have the package
\usepackage{url} % not crucial - just used below for the URL
%%%%%%%%%%%%%%%%%%%%%%%%%%%%%%%%%%%%%%%%%%%%%%%%%%%%%%%%%%%%%%%%%%%%%%%

%%%%% Author package list and commands %%%%%%%%%%%%%%%%%%%%%%%%%%%%%%%%%%%%%%%%%%%%%
%%%%% Here are some examples %%%%%%%%%%%%%%
%	\usepackage{amsfonts, amsthm, latexsym, amssymb}
%	\usepackage{lineno}
%	\newcommand{\mb}{\mathbf}
%%%%%%%%%%%%%%%%%%%%%%%%%%%%%%%%%%%%%%%%%%%%%%%%%%%%%%%%%%%%%%%%%%%%%%%%%%%%%%

\begin{document}
	
		%%%%%%%%%%%%%%%%%%%%%%%%%%%%%%%%%%%%%%%%%%%%%%%%%%%%%%%%%%%%%%%%%%%%%%%%%%%%%%
	\def\spacingset#1{\renewcommand{\baselinestretch}%
		{#1}\small\normalsize} \spacingset{1}
	%%%%%%%%%%%%%%%%%%%%%%%%%%%%%%%%%%%%%%%%%%%%%%%%%%%%%%%%%%%%%%%%%%%%%%%%%%%%%%
	
	\if0\blind
	{
		\title{\bf Simulation-guided Data Assimilation for Kinetics Modeling of Potassium Channel Isoforms in Mouse Cardiomyocytes}
		\author{Haedong Kim $^a$, Hui Yang $^a$, Andrew R. Ednie $^b$, and Eric S. Bennett $^b$ \\
		$^a$ Department of Industrial and Manufacturing Engineering, \\
		The Pennsylvania State University, State College, PA USA \\
        $^b$ Department of Neuroscience, Cell Biology, and Physiology, \\
        Wright State University, City, OH USA}
		\date{}
		\maketitle
	} \fi
	
	\if1\blind
	{
        \title{\bf \emph{IISE Transactions} \LaTeX \ Template}
		\author{Author information is purposely removed for double-blind review}
		
\bigskip
		\bigskip
		\bigskip
		\begin{center}
			{\LARGE\bf \emph{IISE Transactions} \LaTeX \ Template}
		\end{center}
		\medskip
	} \fi
	\bigskip
		
\begin{abstract}
Aberrant activities of voltage-gated potassium (K\textsubscript{v}) channels can cause fatal heart disease such as long QT syndrome. In our recent investigation on a disease-related perturbation, reduced glycosylation in cardiomyocytes, significant reduction in K\textsuperscript{+} currents was observed. Among major cardiac voltage-gated ion channels (VGIC), K\textsubscript{v} channels have a distinctive feature: various isoforms generate their own unique currents contributing to different phases of repolarization of cardiac action potential (AP). On the other hand, the dominant isoform is responsible for Na\textsuperscript{+} and Ca\textsuperscript{2+} currents. Because only the sum of these K\textsuperscript{+} currents (I\textsubscript{Ksum}) can be recorded in experiments, data analytical methods are necessary to estimate the properties of individual K\textsuperscript{+} currents and their effects on the behaviors of the cardiomyocyte. However, traditional approaches estimate only one property, the shape of individual K\textsuperscript{+} currents, from a single I\textsubscript{Ksum} recording rather than analyzing I\textsubscript{Ksum} recordings from the same cell with different conditions. These approaches assume the functional form of the shape of K\textsuperscript{+} currents and search the shape parameters in the current functions that the sum of estimated currents has the best fit with a single I\textsubscript{Ksum} recording. It does not provide information about underlying kinetics of the currents at the cellular level. First, computer models of K\textsuperscript{+} currents are designed with parameters that control the kinetic rates and, in turn, determine the currents. Second, we develop fractional factorial designs to identify the parameters that have significant impacts on the current shapes. Third, we design model calibration as nonlinear optimization that couple the \textit{in-silico} model and \textit{in-vitro} I\textsubscript{Ksum} recordings by minimizing discrepancies between them. Further, we develop a graphical user interface (GUI) application to make the proposed framework accessible for researchers and patricians without engineering backgrounds. We apply the proposed framework to our data, and the experimental results show that the data assimilation method can estimate differences in kinetic rate between the control and experiment groups. This method has strong potential to pave a new way to analyze K\textsuperscript{+} currents.
\end{abstract}
		
\noindent%
{\it Keywords:} Data assimilation, potassium channels, kinetics modeling, model calibration, glycosylation, cardiomyopathy,

%\newpage
\spacingset{1.5} % DON'T change the spacing!

\section{Introduction} \label{s:intro}
Cardiac voltage-gated potassium channels (K\textsubscript{v}) are responsible for repolarizing the action potential (AP) in cardiomyocytes. They are indispensable in the electrical conduction system of the heart. Even modest changes in K\textsubscript{v} activities can significantly affect the AP duration and the QT interval, which lead to fatal heart diseases \citep{ravens2008role}. Glycosylation is a co/posttranslational modification that is critical for protein functions including activities of K\textsubscript{v} and other voltage-gated ion channels (VGICs) \citep{ohtsubo2006glycosylation,ednie2012modulation}. A growing number of studies in diverse directions, ranging from genome-wise to proteomic/glycomic searches, have shown clear links between altered protein glycosylation and heart diseases \citep{yung2004gene,yang2015glycoproteins,miura2016glycomics}. Despite its importance, the underlying mechanism is still largely unknown how changes in glycosylation contribute to heart disease onset and progression. Our \textit{in-vitro} experiments showed that glycosylation regulations modulate VGIC activities and contribute to both electrical and contractile dysfunction \citep{ednie2013sialicNav1,ednie2015sialicKv,ednie2019reduced}. In addition, we verified our experimental results via \textit{in-silico} studies and estimate how changes in the lower level (VGICs) affect the functions in the higher level (cardiomyocytes), which is one of the powers of systematic computer models \citep{du2013silico,du2015statistical,du2017silico,kim2022simulation}. \textbf{Figure~\ref{fig:framework_diagram}} illustrates this feedback loop of studying heart diseases, which will be referred to as the data assimilation framework that combines experimental data with mathematical/computational models.
\begin{figure}[!ht]
    \centering
    \includegraphics{figs/overall_diagram.pdf}
    \caption{Conceptual diagram to illustrate a data assimilation framework.}
    \label{fig:framework_diagram}
\end{figure}

However, a distinctive feature of K\textsubscript{v} raises challenges in both \textit{in-vitro} and \textit{in-silico} studies: K\textsubscript{v} have diverse isoforms, with each of them showing unique biophysical properties and playing different roles in the AP repolarization \citep{nerbonne2005molecular}. In contrast, there are predominant isoforms for Na\textsubscript{v} and Ca\textsubscript{v} \citep{abriel2010cardiac,benitah2010type}. Although it is highly desirable to observe the activities of each K\textsubscript{v} isoform separately, only collective outputs can be measured because of their slightly overlapping voltage-dependence of gating. Specifically, only the sum of K\textsuperscript{+} currents (I\textsubscript{K}) can be recorded in whole-cell voltage-clamp experiments. Typically, curve fitting is used to estimate individual K\textsuperscript{+} currents, in which parameters of the functional form of the currents are searched, thereby the sum of them best describing I\textsubscript{K} recordings \citep{brunet2004heterogeneous}. Then the shape parameters are used to calibrate computer models \citep{kim2022simulation}, or kinetics information is added to the data assimilation process with additional curve fitting if applicable \citep{du2017silico}. However, this traditional approach is limited in its ability to rigorously determine kinetics and gating of the K\textsubscript{v} isoforms in two aspects: 1) This two-step curve-fitting method does not consider the dynamics of the underlying kinetics and current outputs by separating the two processes working together and simplifying them with curve functions, and 2) it does not account for the cellular-level dynamics according to changes in protocols, as one current trace is analyzed separately. These limitations not only hinder \textit{in-silico} modeling results but, in some cases, make it impossible if modeling relies on kinetic data estimated from experimental data \citep{kim2022simulation}. Therefore, there is an urgent need to develop new data assimilation methods and tools that delineate the kinetics of K\textsubscript{v} isoforms for better understanding their roles in heart diseases. 

This paper presents a novel approach to data assimilation for analyzing whole-cell I\textsubscript{K} recordings and modeling the underlying kinetics of K\textsubscript{v} isoforms. A schematic flowchart of the proposed framework is shown in \textbf{Figure~\ref{fig:flow_chart}B} and compared with the current approach in \textbf{Figure~\ref{fig:flow_chart}A}. First, we designed biophysical computer models that simulate activities of ion channels to replace simple current-shape functions in curve fitting. Computer models of ion channels are a system of differential equations consisting of gating variables and kinetic functions that continuously simulate ion channel activities over time. Parameters in kinetic functions control the gating of ion channels and, in turn, determine current outputs. Second, a sensitivity analysis was performed to screen out parameters that marginally influence the model outputs. Third, we developed a model calibration that minimizes discrepancy between model outputs and experimental data generated from multiple protocols. Further, we packaged the suggested framework into a graphical user interface (GUI) application to make the proposed framework more accessible. Experimental results of the proposed method applied to our glycosylation data provide kinetics information and differences between the healthy control and the disease group.
\begin{figure}
    \centering
    \includegraphics{figs/flow_chart.pdf}
    \caption{Flowcharts of (A) the current approach and (B) the proposed method.}
    \label{fig:flow_chart}
\end{figure}

The main advantage of this approach is that it analyzes I\textsubscript{K} recordings based on the biophysical models with considering cellular level dynamics according to changes in protocols for studying heart diseases. Our contributions are summarized as follows:
\begin{itemize}
    \item We modeled the underlying kinetics of K\textsubscript{v} isoforms from whole-cell I\textsubscript{K} recordings based on biophysical principles rather than curve-fitting methods.
    \item We developed a model calibration routine that harnesses multiple protocols simultaneously rather than a single protocol independently, resulting in comprehensive cellular-level modeling.
    \item Our method captures differences in channel kinetics given the disease-related perturbation, such as reduced glycosylation.
\end{itemize}
The proposed data assimilation framework has strong potential to serve as a novel complementary data analytics tool to in-vitro experiments. 

\section{Methods} \label{s:methods}

\subsection{\emph{Computer Models of Potassium Channel Isoforms}} \label{s:methods.1}
Computer models mouse ventricular cardiomyocytes simulate important biophysical properties of ion channels and molecular dynamics in producing the AP. In this paper, the AP is modeled as a differential equation of transmembrane currents and stimulus current I\textsubscript{stim} as below in Equation~\ref{eq:ap} where $C_{m}$ is the membrane capacitance, $t$ is time.
\begin{equation}
    \label{eq:ap}
    \begin{split}
    -C_{m}\frac{dV}{dt} &= \mathrm{I}_{\mathrm{CaL}}+\mathrm{I}_{\mathrm{p}(Ca)}+\mathrm{I}_{\mathrm{NaCa}}+\mathrm{I}_{\mathrm{Cab}}+\mathrm{I}_{\mathrm{Na}}+\mathrm{I}_{\mathrm{Nab}}+\mathrm{I}_{\mathrm{NaK}}+\mathrm{I}_{\mathrm{Cl},Ca} \\
    &+\mathrm{I}_{\mathrm{Kto}}+\mathrm{I}_{\mathrm{Kslow1}}+\mathrm{I}_{\mathrm{Kslow2}}+\mathrm{I}_{\mathrm{Kss}}+\mathrm{I}_{\mathrm{Ks}}+\mathrm{I}_{\mathrm{Kr}}+\mathrm{I}_{\mathrm{K1}}+\mathrm{I}_{\mathrm{stim}}
    \end{split}
\end{equation}
There are 15 transmembrane currents in this AP model: the L-type calcium current (I\textsubscript{CaL}), the calcium pump current (I\textsubscript{p(Ca)}), the Na\textsuperscript{+}/Ca\textsuperscript{2+} exchange current (I\textsubscript{NaCa}), the calcium background current (I\textsubscript{Cab}), the fast Na\textsuperscript{+} current (I\textsubscript{Na}), the background Na\textsuperscript{+} current (I\textsubscript{Nab}), the Na\textsuperscript{+}/K\textsuperscript{+} pump current (I\textsubscript{NaK}), the Ca\textsuperscript{2+}-activated Cl\textsuperscript{-} current (I\textsubscript{Cl,Ca}), the rapidly inactivating transient outward K\textsuperscript{+} current (I\textsubscript{Kto}), the two ultra-rapidly activating delayed rectifier K\textsuperscript{+} currents (I\textsubscript{Kslow1} and I\textsubscript{Kslow2}), the non-inactivating steady-state K\textsuperscript{+} current (I\textsubscript{Kss}), the slow delayed rectifier K\textsuperscript{+} current (I\textsubscript{Ks}), the rapid delayed rectifier K\textsuperscript{+} current (I\textsubscript{Kr}), and the time independent K\textsuperscript{+} current (I\textsubscript{K1}).

The three types of K\textsuperscript{+} currents have larger magnitude compared to the other K\textsuperscript{+} currents: the rapidly inactivating transient outward K\textsuperscript{+} currents (I\textsubscript{Kto,f} and/or I\textsubscript{Kto,s}), the delayed rectifier K\textsuperscript{+} currents (I\textsubscript{Kslow1} and I\textsubscript{Kslow2}), and the non-inactivating steady-state K\textsuperscript{+} current (I\textsubscript{Kss}). The rapidly inactivating transient outward currents have two components I\textsubscript{Kto,f} and I\textsubscript{Kto,s}, but the later is essentially absent in apex cardiomyocytes. We only include I\textsubscript{Kto,f} in I\textsubscript{Kto}, because we focus on apex myocytes in this investigation. However, the transient outward current can be easily modified according to the region of ventricular cardiomyocytes. Therefore, we model K\textsubscript{v} isoforms that conduct these dominant currents: K\textsubscript{v}4.2 (I\textsubscript{Kto}), K\textsubscript{v}1.5 (I\textsubscript{Kslow1}), K\textsubscript{v}2.1 (I\textsubscript{Kslow2}), and K\textsubscript{2P} family (I\textsubscript{Kss}). \textbf{Figure~\ref{fig:kcurrent_example}B} illustrates the shape of the primary K\textsuperscript{+} currents and their contribution to the I\textsubscript{Ksum} given the protocol in \textbf{Figure~\ref{fig:kcurrent_example}A} that applies 0 mV voltage step from holding potential -70 mV from time $t^{(h)}$ to $t^{(e)}$. \textbf{Figure~\ref{fig:kcurrent_example}C} and \textbf{D} show a range of protocols and consequent change in I\textsubscript{Ksum}. Because of their shape, I\textsubscript{Kss} is assumed as a constant and the other currents an exponential function in the traditional data-driven curve-fitting approach. In this approach, K\textsuperscript{+} current can be defined by Equation~\ref{eq:expl_fitting} with the shape parameters such as amplitude $\mathrm{A}_{i}$ and time constant $\tau_{i}$ for $i \in \{\mathrm{Kto}, \mathrm{Kslow1}, \mathrm{Kslow2}, \mathrm{Kss}\}$, and a set of them is estimated that have the best fit with experimental I\textsubscript{Ksum} data.
\begin{figure}[!ht]
    \centering
    \includegraphics{figs/exemplary_kv_currents.pdf}
    \caption{Example of voltage-clamp K\textsuperscript{+} currents. (A) Voltage step of 0 mV from holding potential -70 mV. (B) Dominant K\textsuperscript{+} currents and their contributions to I\textsubscript{Ksum}. (C) Range of protocols from -50 to 50 mV from holding potential -70 mV and (D) consequent K\textsuperscript{+} current traces.}
    \label{fig:kcurrent_example}
\end{figure}
\begin{equation}
    \label{eq:expl_fitting}
    \mathrm{I}_{\mathrm{Ksum}} = \mathrm{A}_{\mathrm{Kto}}e^{-t/\tau_{\mathrm{Kto}}} + \mathrm{A}_{\mathrm{Kslow1}}e^{-t/\tau_{\mathrm{Kslow1}}} + \mathrm{A}_{\mathrm{Kslow2}}e^{-t/\tau_{\mathrm{Kslow2}}} +  \mathrm{A}_{\mathrm{Kss}}
\end{equation}

Unlike the simple current functions, computer models of K\textsubscript{v} isoforms can simulate detailed gating kinetics. We developed them with Hodgkin-Huxley modeling that has two gating variables controlling conductance. For example, K\textsubscript{v} isoforms that conducts current I\textsubscript{K} can be defined by Equation~\ref{eq:hh_ex}
\begin{equation}
    \label{eq:hh_ex}
    \mathrm{I}_{\mathrm{K}} = G_{\mathrm{K}}a^{n}i^{m}(V-E_{\mathrm{K}})
\end{equation}
where $G_{\mathrm{K}}$ is the maximum conductance, $a^{n}$ and $i^{m}$ are the gating variables for $n,m \in \mathbb{N}$, $V$ is the transmembrane potential, and $E_{\mathrm{K}}$ is the K\textsuperscript{+} Nernst potential. $V-E_{\mathrm{K}}$ is the driving force of the ion movement. Important components in this equation are the gating variables $a$ and $i$, representing the fraction of activation and recovery from inactivation of the channel where $a,i \in [0,1]$. These processes are governed by first-order kinetics and its voltage-dependent transition rates $\alpha$ and $\beta$. $\alpha$ is the rate at which a gate in closed state opens, whereas $\beta$ is the rates at which a gate in open state closes. Equation~\ref{eq:gv} shows a schematic relationship of this gating kinetics.
\begin{align}
    \label{eq:gv}
    (1-a)&\xrightleftharpoons[\beta_{a}]{\alpha_{a}}a & (1-i)&\xrightleftharpoons[\beta_{i}]{\alpha_{i}}i
\end{align}

The two biophysical processes can be modeled using differential equations in two ways as follows:
\begin{align}
    \frac{da}{dt} &=\alpha_{a}(1-a)-\beta_{a}a   &\frac{di}{dt} &=\alpha_{i}(1-i)-\beta_{i}i \\
    \frac{da}{dt} &= \frac{a_{\infty}-a}{\tau_{a}}  &\frac{di}{dt} &= \frac{i_{\infty}-i}{\tau_{i}}
\end{align}
where $a_{\infty}$ and $i_{\infty}$ are the steady-state values to which $a$ and $b$ converge; $\tau_{a}$ and $\tau_{i}$ are time constants determine the convergence speed defined by
\begin{align}
    a_{\infty} &= \frac{\alpha_{a}}{\alpha_{a}+\beta_{a}} & i_{\infty} &=  \frac{\alpha_{i}}{\alpha_{i}+\beta_{i}} \\
    \tau_{a} &= \frac{1}{\alpha_{a}+\beta_{a}} & \tau_{i} &= \frac{1}{\alpha_{i}+\beta_{i}}
\end{align}
The steady-state values and time constants can be defined directly by functions of voltage without transition rates in some cases. These voltage-dependent functions, such as transition rates, steady states, or time constants, have parameters that control the behavior of the kinetics of an ion channel.

\subsubsection{In-silico Modeling of K\textsubscript{v}4.2 (I\textsubscript{Kto})}
The rapidly inactivating transient outward current I\textsubscript{Kto}, conducted through K\textsubscript{v}4.2, is characterized by a sharp upstroke during activation and subsequent rapid inactivation. It mainly contributes to the peak at the very beginning of activation in I\textsubscript{Ksum}. I\textsubscript{Kto} is defined by
\begin{align}
    &\mathrm{I}_{\mathrm{Kto}} = G_{\mathrm{Kto}}a_{\mathrm{Kto}}^{3}i_{\mathrm{Kto}}(V-E_{\mathrm{K}}) \\
    &\frac{da_{\mathrm{Kto}}}{dt} = \alpha_{a}(1-a_{\mathrm{Kto}}) - \beta_{a}a_{\mathrm{Kto}} \\
    &\frac{di_{\mathrm{Kto}}}{dt} = \alpha_{i}(1-i_{\mathrm{Kto}}) - \beta_{i}i_{\mathrm{Kto}} \\
    &\alpha_{a} = p_{7}e^{p_{5}(V+p_{1})} \label{eq:ikto_alpha1} \\
    &\beta_{a}= p_{8}e^{-p_6(V+p_{1})} \\
    &\alpha_{i} = \frac{p_{9}e^{-(V+p_{2})/p_{4}}}{1+p_{10}e^{-(V+p_{2}+p_{3})/p_{4}}} \\
    & \beta_{i} = \frac{p_{11}e^{(V+p_{2}+p{3})/p_{4}}}{1+p_{12}e^{(V+p_{2}+p_{3})/p_{4}}} \label{eq:ikto_beta2}
\end{align}
There are two gating variables $a_{\mathrm{Kto}}$ and $i_{\mathrm{Kto}}$ responsible for activation and inactivation, respectively. Their kinetics are governed by transition-rate functions from Equation~\ref{eq:ikto_alpha1} to Equation~\ref{eq:ikto_beta2}. Parameters $p_{i}$ for $i=\{1, 2, \dots, 12\}$ in these equations act like ``knobs'', allowing to control the behavior of the I\textsubscript{Kto} model.

\subsubsection{In-silico Modeling of K\textsubscript{v}1.5 (I\textsubscript{Kslow1}), K\textsubscript{v}2.1 (I\textsubscript{Kslow2}) and K\textsubscript{2P} (I\textsubscript{Kss})}
There are two major delayed rectifier currents in mouse ventricular cardiomyocytes that are very rapidly activating: I\textsubscript{Kslow1} and I\textsubscript{Kslow2}, which are conducted through K\textsubscript{v}1.5 and K\textsubscript{v}2.1, respectively. As shown in \textbf{Figure~\ref{fig:kcurrent_example}}, both rectifier currents inactivate slower and have smaller magnitude than I\textsubscript{Kto}, but I\textsubscript{Kslow2} decays more gradually than I\textsubscript{Kslow1}. The non-inactivating steady-state current, which conducted through K\textsubscript{2P} family, remains constant during the voltage-clamp recording. These three currents collectively address the most part of the decaying phase of I\textsubscript{Ksum}. We assume that I\textsubscript{Kslow1} and I\textsubscript{Kslow2} have the same activation gating variable, and I\textsubscript{Kss} has similar activation behavior with a slightly different rate to keep the model as simple as possible to reduce the structural risk of overfitting.

I\textsubscript{Kslow1} is modeled without transition-rate functions as opposed to I\textsubscript{Kto}. Its gating variables, activation $a_{\mathrm{Kslow1}}$, and inactivation $i_{\mathrm{Kslow1}}$ are defined directly by steady-state ($a_{ss}$ and $i_{ss}$) and time-constant functions ($\tau_{a}^{(1)}$ and $\tau_{i}^{(1)}$) as follows.
\begin{align}
    &\mathrm{I}_{\mathrm{Kslow1}} = G_{\mathrm{Kslow1}}a_{\mathrm{Kslow1}}i_{\mathrm{Kslow1}}(V-E_{\mathrm{K}}) \\
    &\frac{da_{\mathrm{Kslow1}}}{dt} = \frac{a_{ss}-a_{\mathrm{Kslow1}}}{\tau_{a}^{(1)}} \\
    &\frac{di_{\mathrm{Kslow1}}}{dt} = \frac{i_{ss}-i_{\mathrm{Kslow1}}}{\tau_{i}^{(1)}} \\
    &a_{ss} = \frac{1}{1+e^{-(V+p_{1})/p_{4}}} \\
    &i_{ss} = \frac{1}{1+e^{(V+p_{2})/p_{5}}} \\
    &\tau_{a}^{(1)} = \frac{p_{7}}{e^{p_{6}(V+p_{3})} + e^{-p_{6}(V+p_{3})}} + p_{9} \\
    &\tau_{i}^{(1)} = p_{10} - p_{8}i_{ss}
\end{align}

I\textsubscript{Kslow2} has the same activation variable with I\textsubscript{Kslow1}, and the time-constant function for the inactivation $i_{\mathrm{Kslow2}}$ that contains the same steady-state function $i_{ss}$ in I\textsubscript{Kslow1}. As a result of this modeling strategy, mathematical equations of I\textsubscript{Kslow} are given as follows. 
\begin{align}
    &\mathrm{I}_{\mathrm{Kslow2}} = G_{\mathrm{Kslow2}}a_{\mathrm{Kslow2}}i_{\mathrm{Kslow2}}(V-E_{\mathrm{K}}) \\
    &a_{\mathrm{Kslow2}} = a_{\mathrm{Kslow1}} \\
    &\frac{di_{\mathrm{Kslow2}}}{dt} = \frac{i_{ss}-i_{\mathrm{Kslow2}}}{\tau_{i}^{(2)}} \\
    &\tau_{i}^{(2)} = p_{2} - p_{1}i_{ss}
\end{align}

I\textsubscript{Kss} does not have an inactivation variable as it is non-inactivating current. It shares the same steady-state function for activation $a_{ss}$ with the rectifier currents but have a separate time-constant function to address the different activation rate. I\textsubscript{Kss} is modeled as
\begin{align}
    &\mathrm{I}_{\mathrm{Kss}} = G_{\mathrm{Kss}}a_{\mathrm{Kss}}(V-E_{\mathrm{K}}) \\
    &\frac{da_{\mathrm{Kss}}}{dt} = \frac{a_{ss}-a_{\mathrm{Kss}}}{\tau_{a}^{(3)}} \\
    &\tau_{a}^{(3)}= \frac{p_{2}}{e^{p_{1}(V+p_{3}^\prime)}+e^{-p_{1}(V+p_{3}^\prime)}} + p_{3}
\end{align}
Note that $p_{3}^\prime$ is equal to $p_{3}$ in I\textsubscript{Kslow1}.

\subsection{\emph{Data Assimilation and Model Calibration}} \label{s:methods.2}
\textit{Data assimilation} is a method to find the optimal configuration and state of computational models by coupling them with experimental data. Experimental data $\mathcal{D}$ are observations of a real process $\mathcal{R}$ that represents scientific phenomena under investigation. The output of physical experiments $y^{\mathcal{D}}(x)$, given input $x$, inevitably contains errors for various reasons, such as noise in measurement or experimental environment. Suppose $\mathcal{D}$ and $\mathcal{R}$ can be related as follows in Equation~\ref{eq:rd_relation}, where $\epsilon$ is the error term. 
\begin{equation}
    \label{eq:rd_relation}
    y^{\mathcal{R}}(x) = y^{\mathcal{D}}(x) + \epsilon
\end{equation}
Let $y^{\mathcal{M}}(x|\theta)$ denote the output from a computer model $\mathcal{M}$, given parameters $\theta$. Assume that there are discrepancies $\delta(x|\theta)$ for the current states of parameters as follows in Equation~\ref{eq:dm_relation}. 
\begin{align}
    \label{eq:dm_relation}
    y^{\mathcal{D}}(x) &= y^{\mathcal{M}}(x|\theta) + \delta(x|\theta) \text{, so} \\
    y^{\mathcal{R}}(x) &= y^{\mathcal{M}}(x|\theta) + \delta(x|\theta) + \epsilon
\end{align}
Our goal in data assimilation is to calibrate $\theta$ to find the best model states that minimize $\delta(x|\theta)$, while satisfying biophysical constraints. By doing that, \textit{in-silico} models $\mathcal{M}$ are coupled with \textit{in-vitro} experimental data $\mathcal{M}$, which provides two complementary angles to study the real process $\mathcal{R}$.

In this study, $\delta$ is defined by the sum of root-mean-square errors (RMSEs) as in Equation~\ref{eq:rmse}, which measures deviations between experimental data and model predictions from the end of a holding potential $t_i^{(h)}$ to the end of voltage step $t_i^{(e)}$ across a range of protocols, $i=1,2,\cdots,n$.
\begin{equation}
    \label{eq:rmse}
    \delta = \sum_{i=1}^{n} \sqrt{\int_{t_i^{(h)}}^{t_i^{(e)}}\frac{(y_i^{\mathcal{D}}(t) - y_i^{\mathcal{M}}(t|\theta))^2}{t_i^{(e)}-t_i^{(h)}}dt}
\end{equation}
This approach differs from previous studies in two-fold. First, unlike the traditional curve fitting using the exponential model described in Equation~\ref{eq:expl_fitting}, the suggested method includes multiple protocols simultaneously for comprehensive cellular-level modeling. Second, it calibrates computer models directly to I\textsubscript{Ksum} recordings, while the previous studies use statistics estimated from the data \citep{du2015statistical, du2017silico, kim2022simulation}. Calibrating to the current traces themselves raises a challenge in optimization but has advantages when it is hard to estimate statistics from data reliably \citep{kim2022simulation}.

We developed the box-constrained nonlinear optimization routine using the BFGS algorithm with multi-random initial points to minimize $\delta$. Box constraints mean that $\theta$ has a lower and upper bound for each dimension, so solution space is constrained in a hypercube. In this way, the optimization loop can be controlled by users, allowing them to blend their domain knowledge into the modeling. Note that it is possible because the suggested method is based on biophysical models that provide interpretability of $\theta$ in channel kinetics rather than the curve fitting. Global optimum is not guaranteed due to the complexity of our models. We developed the multi-random starting scheme to avoid local optima and find the solution as close to the optimum as possible. Latin hypercube designs and parallel computations are used to sample initial points and run them on multicores to compensate for the increased computational burden. This work is implemented in MATLAB R2022a, and the code is available at xyz.

\subsection{\emph{Sensitivity Analysis and Model Regularization}} \label{s:methods.3}
The principle of parsimony is critical in model calibration not only for enhancing fitting accuracy and preventing overfitting, but also for improving the interpretability of $\theta$. We performed a sensitivity analysis to identify a subset of the parameters that have significant impacts on the model output and only calibrate these sensitive parameters. As illustrated in \textbf{Figure~\ref{fig:anchor_points}}, six \textit{anchor points} were defined that capture representative characteristics of K\textsuperscript{+} current traces in voltage-clamp experiments. Each point represents: a) the current magnitude of 10 ms after applying a voltage step, which measures the activation rate; b) 25\% of the total recording time has elapsed, c) 50\%, and d) 75\%, which collectively estimate the inactivation rate; e) the peak magnitude; and f) the time when current has decayed ($1-e^{-1}$)\% (almost 63\%) from the peak. Feature f will be equal to the total recording time if current does not decline enough as in \textbf{Figure~\ref{fig:anchor_points}C} and \textbf{~\ref{fig:anchor_points}D}. 
\begin{figure}[!ht]
    \centering
    \includegraphics{figs/anchor_ponts.pdf}
    \caption{Illustration of the six anchor points of voltage-clamp K\textsuperscript{+} currents that quantify characteristics of the current shape of (A) I\textsubscript{Kto}, (B) I\textsubscript{Kslow1}, (C) I\textsubscript{Kslow2}, and (D) I\textsubscript{Kss}. All currents are simulated for illustration, and the labels refer to a) the current magnitude 10 ms after voltage is applied, b) 25\% of the total recording time has elapsed, c) 50\%, d) 75\%, e) the peak magnitude, and f) the time when current has decayed ($1-e^{-1}$)\% (almost 63\%) from the peak.}
    \label{fig:anchor_points}
\end{figure}

We developed factorial designs in which parameters varied at two levels, contrasting their effect. The anchor points were evaluated at each design to calculate factorial effects. Then, half-normal plots were drawn based on these contrasts, providing visual inspections of critical parameters that exert the most impact on each shape feature. All the half-normal plots can be found in Supplement A. \textbf{Figure~\ref{fig:calib_param}} shows the selected parameters, highlighted in different colors according to their functional roles in channel kinetics. We categorized the calibration parameters into four classes: The red represents the voltage-threshold parameters and the green voltage slopes, controlling the voltage dependence, the blue scale factors of kinetic functions, and purple time-constant shifters. Note that the voltage-dependence parameters in red and green appear multiple times across different equations. This parsimonious model design is intended to maximize the structural regularization to minimize overfitting.
\begin{figure}[!ht]
    \centering
    \includegraphics{figs/calib_param.pdf}
    \caption{Selected parameters from the sensitivity analysis, highlighted in different colors according to their functional roles in channel kinetics.}
    \label{fig:calib_param}
\end{figure}

\subsection{\emph{Glycosylation and Dilated Cardiomyopathy (DCM)}} \label{s:methods.4}
We applied the proposed framework to our studies on the pathophysiology of electrical signaling by altered glycosylation. Protein glycosylation is one of the most abundant and diverse forms of co/posttranslational modifications that impact essential protein functions, such as modulation of receptor or ion channel activities \citep{ohtsubo2006glycosylation,ednie2012modulation}. A growing number of studies has shown the association between altered glycosylation and heart diseases, such as dilated cardiomyopathy (DCM) and hypertrophic cardiomyopathy \citep{ohtsubo2006glycosylation,ednie2019reduced2}. It is reported that up to 20\% of patients with congenital disorders of glycosylation (CDG), who commonly have modest reductions in protein glycosylation, present with cardiac deficits and idiopathic DCM \citep{marques2017cardiac}. We have investigated how regulated glycosylation contributes to heart failure in the context of electrophysiology. Electrical signaling is orchestrated activities of a variety of ion channels and transporters. Voltage-gated ion channels (VGICs) are heavily glycosylated, with ~30\% of the channel mass consisting of N-/O-linked glycans \citep{ednie2012modulation}.

Glycosylation is a multi-step process and usually ends with sialic acid added. We reported that a saturating, electrostatic effect of negatively charged sialic attached to the terminal of N-/O-glycan branches significantly altered electrical signaling in Na\textsubscript{v} \citep{ednie2013sialicNav1,ednie2015sialicNav2} as well as K\textsubscript{v} \citep{ednie2015sialicKv}. More recently, we showed that preventing hybrid/complex N-glycosylation in mouse cardiomyocytes was sufficient to cause DCM using genetic ablation of the \textit{MGAT1} gene (MGAT1KO model), which encodes a critical glycosyltransferase, GlcNAcT1 \citep{ednie2019reduced2}. MGAT1KO mice developed DCM, deteriorated into heart failure, and 100\% died early, presumably from ventricular arrhythmias leading to sudden cardiac death. To further investigate the role of altered glycosylation and pathogenesis and disease progression of the heart, we conducted whole-cell patch-clamp experiments that showed reductions in N-glycosylation significantly impact electrical signaling in mouse cardiomyocytes \citep{ednie2019reduced}. However, it was difficult to determine channel kinetics rigorously from the \textit{in-vitro} experiments alone, especially for K\textsuperscript{+} current recordings, due to vast diversity of K\textsubscript{v} isoforms and small current magnitude of the MGAT1KO model. Therefore, we validate the suggested framework by modeling K\textsubscript{v} isoforms with our \textit{in-vitro} experimental data. Therefore, we leverage the suggested framework to delineate the channel kinetics of K\textsubscript{v} to provide new descriptive/predictive insights into our \textit{in-vitro} experiments.

\section{Results} \label{s:results}

\subsection{Calibration Results}
first we compared fitting results. 

\subsection{Distributions Estimations}

\subsection{Clustering}
Perhaps numerical analysis of performance, comparison, and demonstration of applicability and impact using real data.

\section{Discuss}
Discuss of the results goes here.

\section{Conclusion}\label{s:conclusion}
A paper ends with a conclusion or summary section.

\if0\blind{
\section*{Acknowledgements}
The authors acknowledge the generous support from the funding agency of XYZ and funding list} \fi

\bibliographystyle{chicago}
\spacingset{1}
\bibliography{IISE-Trans}
	
\end{document}
